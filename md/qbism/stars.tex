\documentclass[]{article}
\date{}
\usepackage{url}
\begin{document}

\section*{Introduction: Quantum Mechanical Tunneling}
Why is the Sun able to produce so much energy over such a long period of time? Physicists tend to answer this question with an explanation that goes something like this:

\begin{quote} The sun is composed of many little parts, including hydrogen atoms. If hydrogen atoms fuse together they yield helium. The difference in mass between the products and reactants is manifested as the release of large amounts of energy. According to quantum theory, hydrogen atoms are able to get close enough together to fuse because they undergo quantum tunneling.
\end{quote}
{\small\url{https://arxiv.org/pdf/1707.02030.pdf}\\
--McQueen 2017 "Is QBism The Future of Quantum Physics?"}\\

For two protons to fuse, they need to come close enough to each other; however, there is a big electrostatic barrier for this to occur. Let's call that $\vee$. In classical physics, we can imagine this problem as a particle in a box. The box prevents our proton from escaping its electrostatic barrier due to insufficient kinetic energy, $E$, to get over the barrier $\vee$. However, if the thickness of the barrier is thin, i.e. $E > \vee$, the particles have some probability of penetrating through the barrier.\\

\section*{Methods: Expanding to QBism from the Schrodinger Equation}
\subsection{Schrodinger Equation}
\underline{Problem 1: Derive the Shrodinger equation for this experiment so that:}
\medskip
{\centering
  
  $\Psi = Ne^{\sqrt{\frac{2m(V-E)}{h^{2}}}}x$\\
  \vspace{5mm}
  $P = e^{\frac{-4a\pi}{h}\sqrt{2m(V-E)}}$\\
  \vspace{5mm}
  ($E<V$)
  \par
}

where
\begin{itemize}
\item $V$ is the potential barrier,
\item $E$ is the kinetic energy possessed by the particle, and
\item $\alpha$ is the thickness of the barrier.
\item $m$ is mass of the particle
\item $h$ is Planks Constant\\
\end{itemize}

For a quantum particle to appreciably tunnel through a barrier three conditions must be met:
\begin{enumerate}
\item The height of the barrier must be finite and the thickness of the barrier should be thin.
\item The potential energy of the barrier exceeds the kinetic energy of the particle ($E<V$). 
\item The particle has wave properties because the wavefunction is able to penetrate through the barrier. This suggests that quantum tunneling only apply to microscopic objects such protons or electrons and does not apply to macroscopic objects.
\end{enumerate}

The probability of an object tunneling through a barrier decreases with the object's increasing mass and decreases with the magnitude of the difference betweent the energy of the object and the energy of the barrier.

\subsection*{QBism Tunneling}
./md/qbism/introducitontoqbism.pdf
\end{document}
